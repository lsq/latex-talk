\section{函数及数据透视表……}

%\begin{frame}{常用函数}
\begin{frame}[fragile]
\frametitle{常用函数}
\pause
\begin{itemize}
	\item<+-> 查找定位函数
	
	\begin{itemize}	
		\item<+-> |vlookup|
		\item<+-> |lookup|、|find|(|search|)组合运用(二分查找原理)
		\begin{itemize}
			\item 多条件查找公式是Lookup函数最经典、最万能的公式。可以归纳为:\\		
			|=Lookup(1,0/((条件1)*(条件2)……条件N),返回值的范围)|。
			\item 逆向查找:|=Lookup(,-find(条件1, 查找区域), 返回值的范围)|
		\end{itemize}
		\item<+-> |index|、|match|运用
		\item<+-> |offset|动态图表
	\end{itemize}
	
	\item<+-> 聚合函数
	\begin{itemize}	
		\item |subtotal|、|sumifs|、|countif|、|sumproduct|
	\end{itemize}

	\item<+-> 其他函数(文本、日期、逻辑、信息)
	\begin{itemize}
		\item<+-> |indirect|、|address|、|cell|
		\item<+-> |iferror|、|if|、|choose|、|transpose|、|left|、|right|、|mid|、|substitute|、|replace|、|text|、|T|、|Value|、|day|、|row|、|columns|、|Mod|、|Product|、|mmult|
		\item<+-> 宏表函数|Evaluate|、|GET.cell|(filename/col/row/address)
	\end{itemize}
	
\end{itemize}
\end{frame}
\begin{frame}{数据透视表}
	\begin{itemize}
		\item<+-> 经典透视表
		\item<+-> 切片器
%		\item<+-> 透视图表(用的少)
		\item<+-> 更多使用方法可以参考\link{https://zhuanlan.zhihu.com/p/74995027}
	\end{itemize}
\end{frame}
%\begin{itemize}
%  \item<+-> 打开终端
%
%    \begin{itemize}
%      \item<+-> \faWindows{}:右键开始菜单、空白处 \kbd{Shift} + 右键、\kbd{Windows} + \kbd{R} \& |cmd|
%      \item<+-> \faLinux{}:\kbd{Ctrl} + \kbd{Alt} + \kbd{T}
%      \item<+-> \faApple{}:\kbd{⌘} + \kbd{Space} 搜索 Terminal、可在 Finder 中添加服务
%    \end{itemize}
%
%  \item<+-> 基本命令:
%
%    \begin{itemize}
%      \item<+-> |cd|、|ls/dir|、|rm/del|、|clear/cls|
%      \item<+-> 选项:|-h|、|--help|、|/?|
%    \end{itemize}
%
%  \item<+-> 其他:
%
%    \begin{itemize}
%      \item<+-> 复制粘贴:\kbd{Ctrl}/\kbd{Shift} + \kbd{Ins}、\kbd{Ctrl}/\kbd{⌘} + \kbd{C}/\kbd{V}、
%      \item<+-> 路径连接符:斜线(|/|)或反斜线(|\|)
%      \item<+-> 换行符:LF(|\n|)或 CRLF(|\r\n|)
%      \item<+-> 结束进程:\kbd{Ctrl} + \kbd{C}
%    \end{itemize} \pause
%
%  \item<+-> \alert{尽量不要用中文;避免空格、特殊符号}
%\end{itemize}
%\end{frame}

%\begin{frame}{编码}
%关于 \LaTeX{} 源文件的编码,我们给出如下结论:\pause
%\begin{alertblock}{编码定理}
%  一般地,在任何场合使用(不带 BOM)的 \alert{UTF\CASE{-}8} 编码均是最优选择.
%\end{alertblock} \pause
%此定理的证明留做习题.
%\end{frame}
